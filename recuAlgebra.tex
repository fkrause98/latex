\documentclass[12pt,twoside,a4paper]{exam}
\footer{}{\thepage}{}
\usepackage{amsmath,mathtools,amsfonts,verbatim,amssymb}
\title{Álgebra I \\ Segundo recuperatorio del Segundo parcial \\ 17/12/2019}
\begin{document}
\date{}
\maketitle

\begin{questions}

\question Hallar el resto de de la división de  $12^{(2^n)}$ por 7 para cada n $\in\mathbb{N}$.
\question Probar que: $|1+iz|=|1-iz|\iff z \in \mathbb{R}$.
\question Sea $\omega\in G_5$ una raíz quinta primitiva de la unidad. Hallar todos 
          los n $\in \mathbb{N}$ tales que:\\ $$\sum_{j=0}^{3n+2}(\overline{\omega^{-11}}
          +\omega^{10}+\omega^{-3}+\omega^{4}+\omega^{14^2}\overline{\omega^{12}})^j
          =\omega + 1 $$
\question Sea $f=X^6 -3X^4 -(2+8i)X^3+24iX+16i$. Hallar todas las raíces complejas
          de $f$, sabiendo que tiene al menos una raíz entera.
\question Hallar $f\in \mathbb{Q}[X]$ de grado mínimo que cumpla las siguientes
          condiciones simultáneamente:
          \begin{itemize}
          \item $f$ comparte una raíz con $X^3-3X^2+7X-5$.
          \item $X+3-\sqrt{2}$.
          \item $1-2i$ es raíz de $f$ y $f'(X-2i)=0$.
          \end{itemize}
          A continuación, factorizar $f \text{ en } \mathbb{C}[X]$, $\mathbb{R}[X]$
          y $\mathbb{Q}[X]$.
\end{questions}
\section*{Solución de 1:}
\begin{itemize}
\item Quiero encontrar $x = r_7((12)^{2^n})$, para esto, uso Fermat. Como 7 es 
      primo y (12:7) = 1 $\implies (12)^{2^n} \equiv (12)^{r_6(2n)} (7).$
\item Recordemos que, dados $a$,$b \in \mathbb{Z}$ y que si $r$ es el resto de
      dividir a por b, $a \equiv r (b)$, $\implies (12)^{r_6(2^n)} \equiv x (7) $
      $\implies (12)^{r_6(2^n)} \equiv x (7)$. Entonces, busco $r_6(2^n)$ para 
      seguir.
\item Sea $y = r_6(2^n)$, se me ocurre encontrar $y$ mediante TCR, ya que (2:3) = 1.\\
 \[
     2^n \equiv y (6) \iff 
     \begin{cases}
        2^n \equiv y (2)\\
        2^n \equiv y (3)\\
     \end{cases}
     \iff
      \begin{cases}
        y \equiv 0 (2)\\
        2^n \equiv y (3)\\
     \end{cases}
\]
    Por Fermat, $2^n \equiv 2^{r_2(n)} (3)$, los posibles restos de n son 0 ó 1,
    que se corresponden con n par o impar.\\
\item Asumo n par: $\implies 2^{r_2(n)} \equiv 2^0 \equiv  1 (3) \implies y \equiv
    1 (3)$, entonces, se tiene que: 
\[
     \begin{cases}
     y \equiv 0 (2) \iff y=2q \text{, para algún } q \in \mathbb{Z} \\
     y \equiv 1 (3) \iff 2q \equiv 1 (3) \iff -q \equiv 1 (3) \iff q \equiv -1 (3)
     \iff q\equiv 2 (3)\\ 
     \end{cases}
\]
     Así, $y=2q=2(3k+2)=6k+4 \equiv 4 (4)\text{, para algún } k \in \mathbb{Z}$ \\
     Entonces, $2^n \equiv 4 (6) \implies (12)^{r_6(2^n)} \equiv 12^4 \equiv 2 (7) 
    \implies r_7(12^{2^n})=2$, sé que es 2 ya que el resto es único. 
\item Asumo n impar: se llega a que, con un proceso similar al anterior,  $2^n 
      \equiv 2 (6)$  y que $r_7(12)^{(2^n)}=4$.
\end{itemize}
\section*{Solución de 2:} 

\begin{itemize}
\item Quiero ver que: $$|1+iz|=|1-iz| \iff z \in \mathbb{R}$$
\item $\implies$), qvq: $|1+iz|=|1-iz| \implies z \in \mathbb{R}$\\ 
\begin{align*}
  &\text{Asumo que z} \in \mathbb{C} \text{ con forma binomial } z=a+bi \text{ con 
   }a,b \in \mathbb{R} \text{ tal que } |1+iz|=|1-iz|\\
  &\implies |1+i(a+bi)|=|1-i(a+bi)|\\
  &\implies |1+ai-b|=|1-ai+b|\\
  &\implies |(1-b)+ai|=|(1+b)-ai|\\
  &\implies \sqrt{(1-b)^2+a^2}=\sqrt{(1+b)^2+a^2} \\
  &\implies (1-b)^2+a^2=(1+b)^2+a^2\\
  &\implies (1-b)^2=(1+b)^2\\
  &\implies 1-2b+b^2=1+2b+b^2\\
  &\implies -b=b \\
  &\implies  b=0 \\
  &\implies z=a+0i\\ 
  &\implies z \in \mathbb{R} 
\end{align*}
\item $\impliedby$), qvq: $z \in \mathbb{R} \implies |1+iz|=|1-iz|$
\begin{flalign*}
&z=z  \iff 1+z^2=1+z^2&\\
&\implies \sqrt{1+z^2}=\sqrt{1+z^2}&\\
&\implies \sqrt{1+z^2}=\sqrt{1+(-z^2)}& \\
&\implies |1+iz| = |1-iz|&
\end{flalign*}
$\blacksquare$
\end{itemize}
\section*{Solución de 3:}

\begin{itemize}
\item Sé que $\omega \in G_5*$, esto me dice que si tengo $\omega^n, n \in \mathbb{Z} 
  \implies \omega^n=w^{r_5(n)}\text{, además, } \overline{\omega}=\omega^{-1}$ que voy a usar para
      reescribir la expresión de la sumatoria a algo más amigable.\\
      $\overline{\omega^{-11}}=\omega$, $\omega^{10}=1$, $\omega^{-3}=\omega^{2}$
      , $\omega^{14^2}=\omega^{196}=\omega$, $\overline{\omega}^{12}=\omega^{-12}=w^3.$
      \\Por lo tanto: 
\begin{align*}
      \overline{\omega^{-11}}+\omega^{10}+\omega^{-3}+\omega^{4}+
      \omega^{14^2}+\overline{\omega^{12}}&= 1+\omega+\omega^2+\omega^3+\omega^4
      +\omega  \\&=(\sum_{j=0}^4\omega^j)+w =\frac{\omega^5-1}
      {\omega-1} +\omega = \omega  \\
\end{align*}
    Notar que $\omega-1\neq0$, ya que $\omega \in G_5*$.
\item Así,  $$\sum_{j=0}^{3n+2}(\overline{\omega^{-11}}+\omega^{10}+\omega^{-3}
      +\omega^{4}+\omega^{14^2}\overline{\omega^{12}})^j=\sum_{j=0}^{3n+2}\omega^j=
      \frac{\omega^{3n+2+1}-1}{\omega-1}= \frac{\omega^{3(n+1)}-1}{\omega-1}$$
\item Ahora, quiero ver para cuáles $n \in \mathbb{N}$ se cumple que: $$
      \frac{\omega^{3(n+1)}-1} {\omega-1}=\omega+1$$
      \begin{align*}
       \frac{\omega^{3(n+1)}-1}{\omega-1}=\omega+1 & \iff \omega^{3(n+1)}-1 =
       (\omega+1)(\omega-1) \\ 
       &\iff \omega^{3(n+1)}-1 = \omega ^2-1 \\
       &\iff  \omega^{3(n+1)} = \omega^2 \\ 
       &\iff 3(n+1) \equiv 2 (5) \\
       &\iff n \equiv 3 (5) \\
       \end{align*}
      Entonces, los $ n \in \mathbb{N}$ que cumplen son $n \equiv 3 (5)$
\section*{Solución de 4:}
\item Me piden todas las raíces en $\mathbb{C}$ de $f$.
\item Reescribo $f$: \\
      $f(x)=X^6-3X^4-2X^3-8iX^3+24iX+16i$ \\
      $f(x)=X^6-3X^4-2X^3+8i(-X^3+3X+2)$  \\
      $f(x)=X^3(X^3-3X+2)+8iX^3(-X^3+3X+2)$ \\
      $f(x)=(X^3+8iX^3) (-X^3+3X+2)$\\
\item Sea $g(x)=-x^3+3x+2 $, busco sus raíces:\\
  Como $g(x) \in \mathbb{Z}[X]$, por Gauss, las posibles raíces de $g(x)$ en $\mathbb{Q}$ son $\{\pm 2,\pm 1\}$.
      De estas, 2 resulta ser raíz de $g(x)$.\\ 2 es raíz de $g(x) \iff X-2| g(x)$.\\
      Si divido a  $g(x)$ por  $X-2$, obtengo que $g(x)=(X-2)(-X^2-2x-1)=
      -(X-2) (+X^2+2X+1)= -(X-2) (+X+1)^2$\\ Las raíces de $g(x)$ son, entonces, 2 y -1, -1 siendo doble.
\item Sea  $j(x)=-X^3+8i$, sus raíces son las soluciones de $-X^3+8i=0 \iff 8i=X^3$.
      2 números complejos son iguales si comparten módulo y argumento, entonces
      veo para cuáles $X \in \mathbb{C}$ se cumplen ambos.\\
      Módulo: $|X|^3=|8i| \iff |X^3|=\sqrt(8^2) \iff |X^3|=8 \iff |X|=2$
\item Argumento: veo para cuáles $X$ se cumple que $\arg(X^3)=\arg(8i)$\\
      $\arg(8i)=\frac{\pi}{2}$, ya que es imaginario puro y positivo  \\
      Planteo $\arg(X^3)=\frac{\pi}{2}$, por De Moivre, $\arg(X^3)=3\arg(X)+2k\pi$
      donde $k \in \mathbb{Z}$ tal que $3\arg(X)+2k\pi \in [0,2\pi)$\\
      $\implies 3\arg(X)+2k\pi =  \frac{\pi}{2 } \implies 3\arg(X)=-2k\pi+\frac{\pi}{2 }$ 
      $\implies \arg(X)=\frac{\pi}{3}(\frac{1}{2}-2k)$\\ Ahora, hay que ver para
      cuáles $k \in \mathbb{Z} $, $\arg(X) \in [0,2\pi)$:\\
      $0 \leq \frac{\pi}{3}(\frac{1}{2}-2k) < 2\pi \iff 0\leq 1-2k < 6 \iff -1 \leq -2k < 5
      \\ \iff \frac{1}{2} \geq k < - \frac{ 5}{2} \implies k \in \{0,-1,-2\}$.\\
      Entonces, las soluciones de $8i=X^3$ son $a_k=2e^{i\frac{\pi}{3}(1-2k)}$,
      donde $k \in \{0,-1,-2\}$.
\item Así, todas las raíces de $f$ en $\mathbb{C}$ son $\{-1,2,a_k\}$.
\section*{Solución de 5}
\item Enumero cada condición:
    \begin{enumerate}
      \item $f$ tiene una raíz de $g$.
      \item $X+(-\sqrt{2}+3) | f$.
      \item $f(1-2i)=0$ y $f^{'}(1-2i)=0$.
      \item $f \in \mathbb{Q}[X]$, y además, f tiene grado mínimo.
    \end{enumerate}

    \begin{enumerate}
      \item Busco las raíces de $g$, que como pertenece a $\mathbb{Z}[X]$, veo si
            las puedo encontrar mediante Gauss. Las posibles raíces de $g$ en 
            $\mathbb{Q}$ son  $\{\pm 5, \pm 1\}$, de estas, 1 resulta ser raíz
            de $g$. Si divido a $g$ por $X-1$, se tiene que $g=(X-1)(X^2-2X+5)$.
            El discriminante de $(X^2-2X+5)$ es -36, es decir, tiene raíces en
            $\mathbb{C}-\mathbb{R}$. Quiero que $f$ comparta una raíz de $g$,
            y que sea de grado mínimo, supongamos que la raíz que comparten es
            una de las raíces de $(X^2-2X+5)$, como se tiene que cumplir que
            $ f \in \mathbb{Q}[X]$, si tomo una de esas raíces, necesito su 
            conjugado, es decir, en este caso $f$ tiene 2 raíces (y mayor grado
            por el Teorema Fundamental del Álgebra), pero si la raíz que comparten 
            $f$ y $g$ es 1, solamente tengo una raíz, y por consecuente, menor 
            grado que elegir otra raíz de $g$.
      \item $X+(-\sqrt{2}+3) | f \iff \sqrt{2}-3$ es raíz de $f$, y como $ f \in 
            \mathbb{Q}[X]$, y ya que $\sqrt{2} \notin \mathbb{Q} \implies  \sqrt{2}+3$
            también es raíz de $f \iff X+(-\sqrt{2}-3) | f$, además, al ser 
            $X+(-\sqrt{2}-3)$ y $X+(-\sqrt{2}+3)$ coprimos (ambos grado 1 con 
            distinta raíz) $\implies (X+(-\sqrt{2}+3))(X+(-\sqrt{2}-3))|f$.
      \item Necesito que $f(1-2i)=0$ y $f^{'}(1-2i)=0 \iff X-(1+2i)|f$ y $X-(1+2i)|f$  
            $\implies (X-(1+2i))^2|f \implies (X-(\overline{(1+2i)})^2|f$, esta 
            última implicación es por la condición 4.
        \end{enumerate}
      \item Entonces, un $f$ que cumple es: \\
            $f(x)=(X-1)(X+(-\sqrt{2}+3))(X+(-\sqrt{2}-3))(X-(1+2i))^2(X-(1+2i))^2$\\
            Que al ser todas expresiones de grado 1, son irreducibles en $\mathbb{C}[X]$.\\
      \item La expresión irreducible de $f$ en  $\mathbb{R}[X]$ es : \\
            $f(x)=(X-1)(X+(-\sqrt{2}+3))(X+(-\sqrt{2}-3))(X^2-2X+4)^2$\\
            Que son expresiones de grado 1, y un polinomio de grado 2
            con discriminante ne-gativo, por lo que son irreducibles en $\mathbb{R}[X]$.\\
      \item En $\mathbb{Q}[X]$,la expresión irreducible de$f$ es:\\
            $f(x)=(X-1)(X^2+3X+5)(X^2-2X+4)^2$\\
            La primera y última expresión son irreducibles por la misma razón que en 
            $\mathbb{R}[X]$, la segunda, de ser reducible en $\mathbb{Q}[X]$, deberían
            existir $p$ y $q$ en $\mathbb{Q}[X]$ tales que su producto sea igual
            a $(X^2+3X+5)$, pero ya vimos que  $p$ y $q$ están en $\mathbb{R}[X]$
            y sé que la factorización en irreducibles es única.
      \item Por último, veamos que el $f$ al que llegué, es de grado mínimo:\\
            Supongamos que $\exists$ $w \in \mathbb{Q}[X]$, que cumple las condiciones
            1 (con raíz 1), 2 y 3, y que además cumple $gr(w)<gr(f)=7$:\\
            Por 1, $w$ tiene 1 raíz \\
            Por 2, $w$ tiene 2 raíces \\
            Por 3, $w$ tiene 4 raíces \\
            Por el Teorema Fundamental del Álgebra, $w$ tiene grado 7, pero
            habíamos supuesto $gr(w) < 7 $, ¡Absurdo! $\implies f$ es de grado mínimo.

\end{itemize}
\end{document}
